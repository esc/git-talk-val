\documentclass{beamer}

\usepackage[utf8]{inputenc}
\usepackage[T1]{fontenc}
\usepackage{color}
\usepackage{textcomp}
\usepackage{graphicx}

\usepackage{beamerthemesplit}
\usetheme{esc}
\author{Valentin H\"anel}
%\email{valentin.haenel@gmx.de}
\institute{Bernstein Center for Computational Neuroscience Berlin}

\title{Git}
\definecolor{gray}{rgb}{0.10,0.10,0.10}
\date{April 2, 2009}

\begin{document}
%\frame{\titlepage}

\begin{frame}
	\includegraphics[scale=0.05]{BCCN_logo_berlin.jpg}
	\titlepage
\end{frame}

\input{slides.wiki.tex}


\begin{frame}[fragile]
\frametitle{Inhalt von .git}

git speichert alles in dem Verzeichniss {\tt.git/} auch bekannt als \textbf{git
directory}

\begin{verbatim}
    % tree -L 1  .git/
    .
    |-- HEAD            # der gegenwärtige HEAD
    |-- config          # Konfiguration (user, email)
    |-- index           # der index
    |-- objects         # die eigentlichen Objekte

\end{verbatim}

\end{frame}


\end{document}
